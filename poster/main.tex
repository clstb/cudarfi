\documentclass[a0paper,landscape]{baposter}

\usepackage{url}			       % For \url
\usepackage{epstopdf}	       % Included EPS files automatically converted to PDF to include with pdflatex
\usepackage{multicol}        % Multi Columns

\usepackage{amsmath,amssymb} % math
\usepackage{natbib}
\usepackage{graphicx}

%%%%%%%%%%%%%%%%%%%%%%%%%%%%%%%%%%%%%%%%%%%%%%%%%%%%%%%%%%%%%%%%%%%%%%%%%%%%%%%%
%%% Utility functions %%%%%%%%%%%%%%%%%%%%%%%%%%%%%%%%%%%%%%%%%%%%%%%%%%%%%%%%%%
%%%%%%%%%%%%%%%%%%%%%%%%%%%%%%%%%%%%%%%%%%%%%%%%%%%%%%%%%%%%%%%%%%%%%%%%%%%%%%%%

%%% Save space in lists. Use this after the opening of the list %%%%%%%%%%%%%%%%
\renewcommand{\vec}[1]{\bm{#1}}
\newcommand{\vnabla}{\vec{\nabla}}

\renewcommand{\d}[1]{\text{d} #1}
\newcommand{\dxx}{\,\text{d}\vec{x}}
\newcommand{\dx}{\,\text{d}x}

\newcommand{\diff}[2]{\frac{\text{d}#1}{\text{d}#2}}
\newcommand{\idiff}[2]{\text{d}#1 / \text{d}#2}
\newcommand{\pdiff}[2]{\frac{\partial #1}{\partial #2}}
\newcommand{\pdifff}[2]{\frac{\partial^2 #1}{\partial #2^2}}
\newcommand{\ipdiff}[2]{\partial #1 / \partial #2}
\newcommand{\vdiff}[2]{\frac{\delta #1}{\delta #2}}
\newcommand{\ivdiff}[2]{\delta #1 / \delta #2}

%%%%%%%%%%%%%%%%%%%%%%%%%%%%%%%%%%%%%%%%%%%%%%%%%%%%%%%%%%%%%%%%%%%%%%%%%%%%%%%
%%% Document Start %%%%%%%%%%%%%%%%%%%%%%%%%%%%%%%%%%%%%%%%%%%%%%%%%%%%%%%%%%%%
%%%%%%%%%%%%%%%%%%%%%%%%%%%%%%%%%%%%%%%%%%%%%%%%%%%%%%%%%%%%%%%%%%%%%%%%%%%%%%%

\begin{document}
\typeout{Poster rendering started}

%%% COLOR DEFINITIONS %%%%%%%%%%%%%%%%%%%%%%%%%%%%%%%%%%%%%%%%%%%%%%%%%%%%%%%%%
\definecolor{uniblue}{RGB}{0,66,137}
\definecolor{tured}{RGB}{197,14,31}

%%% NICE RULER %%%%%%%%%%%%%%%%%%%%%%%%%%%%%%%%%%%%%%%%%%%%%%%%%%%%%%%%%%%%%%%%
\renewcommand{\hhrule}{\bigskip{\color{tured}\rule{\textwidth}{1pt}}\\}
%%% ITEMIZE %%%%%%%%%%%%%%%%%%%%%%%%%%%%%%%%%%%%%%%%%%%%%%%%%%%%%%%%%%%%%%%%%%%
\renewcommand{\labelitemi}{\color{tured}\scriptsize$\blacksquare$}

%%% General Poster Settings %%%%%%%%%%%%%%%%%%%%%%%%%%%%%%%%%%%%%%%%%%%%%%%%%%%
%%%%%% Eye Catcher, Title, Authors and University Images %%%%%%%%%%%%%%%%%%%%%%
\begin{poster}{
  columns=3,
	grid=false,
	borderColor=tured,
	headerColorOne=tured,
	headerColorTwo=tured,
	headerFontColor=white,
  headerheight=10em,
	boxColorOne=white,
  boxpadding=1em,
	headershape=rectangle,
	headerfont=\Large\textsf,
	textborder=none,
	background=shadetb,
  bgColorOne=white,
  bgColorTwo=white,
	headerborder=open,
  boxshade=plain,
  eyecatcher=false
}
%%% Eye Catcher %%%%%%%%%%%%%%%%%%%%%%%%%%%%%%%%%%%%%%%%%%%%%%%%%%%%%%%%%%%%%%%
{
}
%%% Title %%%%%%%%%%%%%%%%%%%%%%%%%%%%%%%%%%%%%%%%%%%%%%%%%%%%%%%%%%%%%%%%%%%%%
{\Huge CUDARFI: CUDA-Accelerated ARFI Beamforming}
%%% Authors %%%%%%%%%%%%%%%%%%%%%%%%%%%%%%%%%%%%%%%%%%%%%%%%%%%%%%%%%%%%%%%%%%%
{
  \vspace{1em}
  Claas Störtenbecker\\
	{\Large claas.stoertenbecker@gmail.com}\\
  {\Large \url{https://github.com/clstb/cudarfi}}
}
%%% Logo %%%%%%%%%%%%%%%%%%%%%%%%%%%%%%%%%%%%%%%%%%%%%%%%%%%%%%%%%%%%%%%%%%%%%%
{\begin{minipage}{20.0em}
    \includegraphics[height=9em]{tu_berlin_logo}
  \end{minipage}
}

%%% Background %%%%%%%%%%%%%%%%%%%%%%%%%%%%%%%%%%%%%%%%%%%%%%%%%%%%%%%%%%%%%%%%%%
\headerbox{Background}{name=abstract,column=0,row=0}{
  Acoustic Radiation Force Impulse (ARFI) imaging uses focused ultrasound to induce localized tissue displacement. By tracking the resulting shear waves, ARFI provides quantitative information about tissue stiffness, enabling non-invasive assessment of conditions like liver fibrosis or breast lesions.
}

%%% Box 1 %%%%%%%%%%%%%%%%%%%%%%%%%%%%%%%%%%%%%%%%%%%%%%%%%%%%%%%%%%%%%%%%%%%%%
%%% Experimental Setup %%%%%%%%%%%%%%%%%%%%%%%%%%%%%%%%%%%%%%%%%%%%%%%%%%%%%%%%
\headerbox{Experimental Setup}{name=setup,column=1,row=0}{
  The setup utilizes a CIRS phantom imaged with a Verasonics Vantage system.
  
  \begin{center}
    \begin{minipage}{0.48\linewidth}
      \centering
      \includegraphics[width=\linewidth]{schema-1.png}\\
      \footnotesize{Schema 1: Geometry}
    \end{minipage}
    \hfill
    \begin{minipage}{0.48\linewidth}
      \centering
      \includegraphics[width=\linewidth]{schema-3.png}\\
      \footnotesize{Schema 3: Memory Access}
    \end{minipage}
  \end{center}
  
  \vspace{1em}
  \begin{center}
    \begin{minipage}{0.48\linewidth}
      \centering
      \includegraphics[width=\linewidth]{schema-2.png}\\
      \footnotesize{Schema 2: Parallelism}
    \end{minipage}
    \hfill
    \begin{minipage}{0.48\linewidth}
      \centering
      \includegraphics[width=\linewidth]{cirs-phantom.png}\\
      \footnotesize{CIRS Phantom}
    \end{minipage}
  \end{center}
}

%%% Algorithm & Optimization %%%%%%%%%%%%%%%%%%%%%%%%%%%%%%%%%%%%%%%%%%%%%%%%%%
\headerbox{Algorithm \& Optimization}{name=algo,column=0,below=abstract,above=bottom}{
  \raggedright
  \textbf{1. Hilbert Transform (FFTW)}\\
  RF signals are real-valued; converting them to complex analytic signals enables envelope detection and phase tracking. The analytic signal is: $s_a(t) = s(t) + jH\{s(t)\}$, computed via FFT by zeroing negative frequencies and applying IFFT.
  
  \vspace{0.5em}
  \textbf{2. Delay-and-Sum Beamforming (CUDA)}\\
  For each output pixel at position $(x, z)$, signals from all $N$ transducer elements are coherently summed after applying appropriate time delays:
  \begin{equation*}
    I(x, z) = \sum_{i=1}^{N} S_i\left(\frac{\sqrt{(x - x_i)^2 + (z - z_i)^2}}{c} + t_0\right)
  \end{equation*}
  Here, $(x_i, z_i)$ is the position of element $i$, $c$ is sound speed, and $t_0$ is the initial time offset. Linear interpolation provides sub-sample precision.
  
  \vspace{0.5em}
  \textbf{3. Kasai Autocorrelation (Python)}\\
  Tissue displacement $d$ is estimated from the phase shift between consecutive beamformed frames $I_n$ and $I_{n-1}$:
  \begin{equation*}
    d = \frac{c}{4\pi f_0} \angle(I_n \cdot I_{n-1}^*)
  \end{equation*}
  where $f_0$ is the center frequency. A spatial averaging filter smooths the noisy phase estimates.
  
  \vspace{0.5em}
  \textbf{CUDA Optimizations:}
  \begin{itemize}
    \item \textbf{Texture Cache}: \texttt{\_\_ldg()} for read-only cache.
    \item \textbf{Constant Memory}: Probe geometry broadcast.
    \item \mbox{\textbf{Fast~Math}}: \mbox{(\texttt{\_\_fsqrt\_rn}, \texttt{\_\_fmaf\_rn})}.
    \item \textbf{Launch Bounds}: \mbox{\texttt{\_\_launch\_bounds\_\_(256)}}.
    \item \textbf{Precomputed Inverses}: Avoid division by precalculating $1/c$.
  \end{itemize}
}

%%% Results %%%%%%%%%%%%%%%%%%%%%%%%%%%%%%%%%%%%%%%%%%%%%%%%%%%%%%%%%%%%%%%%%%%
\headerbox{Results}{name=results,column=2,row=0}{
  \begin{center}
    \includegraphics[width=0.8\linewidth]{b_mode.png}\\
    \small{Reconstructed B-Mode Image}
  \end{center}
  
  \vspace{0.5em}
  \textbf{Wave Propagation}\\
  \includegraphics[width=0.23\linewidth]{wave_1.png}
  \includegraphics[width=0.23\linewidth]{wave_2.png}
  \includegraphics[width=0.23\linewidth]{wave_3.png}
  \includegraphics[width=0.23\linewidth]{wave_4.png}
  
  \vspace{1em}
  \textbf{Performance Benchmarks} (RTX 3060 vs Ryzen 7800X3D)\\
  \begin{center}
  \begin{tabular}{l l r}
    \textbf{Implementation} & \textbf{Time} & \textbf{Speedup} \\
    \hline
    CPU (OpenMP) & 1.4375 s & 1.00$\times$ \\
    CUDA Naive & 0.8025 s & 1.79$\times$ \\
    \textbf{CUDA Optimized} & \textbf{0.6500 s} & \textbf{2.21$\times$} \\
  \end{tabular}
  \end{center}
}

%%% References %%%%%%%%%%%%%%%%%%%%%%%%%%%%%%%%%%%%%%%%%%%%%%%%%%%%%%%%%%%%%%%%
\headerbox{References}{name=refs,column=2,below=results,above=bottom}{
\begingroup
\tiny
\renewcommand{\section}[2]{}%
\nocite{*}
\bibliography{references}
\bibliographystyle{plain}
\endgroup
}

\end{poster}
\end{document}
